% !TeX root = mos-en.tex

%%%%%%%%%%%%%%%%%%%%%%%%%%%%%%%%%%%%%%%%%%%%%%%%%%%%%%%%%%%%%

\begin{center}
\textbf{\LARGE The Monty Hall Problem}
\end{center}

\addcontentsline{toc}{section}{\large The Monty Hall Problme}

%%%%%%%%%%%%%%%%%%%%%%%%%%%%%%%%%%%%%%%%%%%%%%%%%%%%%%%%%%%%%

The prisoner's dilemma (Problem~13) is very similar to the famous Monty Hall problem \cite{carlton,jason}. This section explains the relation between the two problems and claims that understanding the problems and their solutions is facilitated if they are interpreted in the bayesian interpretation of probability and not the frequentist interpretation according to which Mosteller's book is written.

\textbf{The rules of the game:} In a television game show the contestant stands before three doors. Behind each door is one of three prizes: a car and two goats. The position of each prize is determined randomly with uniform distribution and the host knows the location of each prize. The contestant chooses one door and before the doors are opened and he discovers which prize he won, the host opens one of the doors with a goat. There are two possibilities:
\begin{itemize}
\item If the contestant chose the door with the car, the host opens one of the two doors with goats and the choice is random with uniform distribution.
\item If the contestant chose a door with a goat, the host opens the door with the second goat. 
\end{itemize}
After the door is opened the contestant decides whether to stay with his original choice or to change it and choose the other unopened door. The host opens the door that was chosen and the contestant wins the prize behind that door.
What is the probability that the contestant wins the car if he stays with his original choice and what is the probability that the contestant wins the car if he changes his choice?

\textbf{An incorrect solution and a correct solution:} Many people claim that there is no difference between the probabilities because the car is behind one of the two remaining closed doors. This solution is not correct because these aren't two independent trials, that is, the host does not toss a coin to decide where to place the car and where to place the remaining goat. Any of the solutions for the prisoner's dilemma will give the correct solution to the Monty Hall problem, which is that with probability $2/3$ the car is  behind the other door that wasn't chosen, so the contestant should choose the other door. Here is a list of events and you can draw a tree like in Figure~\ref{f.pp}.
\begin{description}
\item[$e_1$:] The contestant chooses the door with the car and the host opens the door with goat-1.
\item[$e_2$:] The contestant chooses the door with the car and the host opens the door with goat-2.
\item[$e_3$:] The contestant chooses the door with goat-1 and the host opens the door with goat-2.
\item[$e_4$:] The contestant chooses the door with goat-2 and the host opens the door with goat-1.
\end{description}

%%%%%%%%%%%%%%%%%%%%%%%%%%%%%%%%%%%%%%%%%%%%%%%%%%

\textbf{The bayesian interpretation:}

According to the frequentist interpretation it is clear that if the contestant will always changes his choice he will win in two-thirds of the games (see the results of a simulation). However, in order to formulate a strategy for the contestant, it is preferable to discuss the problem within the bayesian interpretation where probability is the level of \textit{belief}. The game concerns a specific contestant who plays the game on a specific date. The additional information from opening a door with the goat can increase the contestant's belief in the position of the car, decrease it or it make no change at all. The usual wordings of the problem do not ask about probabilities, but rather about a strategy: should the contestant change his decision or not? Clearly, the intention is for a bayesian belief.

Suppose that we modify the rules of the game so that the contestant \textit{cannot} change his choice of doors. The probabilities don't change: the probability to win a car remains $1/3$ and the probability to win a goat remains $2/3$. It doesn't matter what door the host opens because the contestant can't change anything. The original problem is different. The probabilities don't change in the sense that the probability that the car is behind the door originally chosen is $1/3$, and the probability that it is behind one of the other two doors is $2/3$. But now the probability of $2/3$ is sort of ``concentrated'' behind the door that the host did not open, and, therefore, according to the bayesian interpretation, the contestant's belief that the care is behind the other door is doubled from $1/3$ to $2/3$ and he should change his choice.

It is also preferable to interpret the prisoner's dilemma within the bayesian interpretation. It concerns a specific set of prisoners on a specific date. The question asks how $A$'s \textit{belief} whether he will be released or not is changed by the additional information that $B$ will be released. The solutions show that there is no justification for change his belief. The bayesian interpretation is implicit in Mosteller's wording of the problem:
\begin{quote}
[$A$] thinks that if the warden says ``$B$ will be released,'' his own chances have now gone down to $1/2$ \ldots{} \cite[p.~4]{fifty}.
\end{quote}
From the words ``thinks'' and ``own'' it is clear that the intention is a bayesian interpretation, because in the frequentist interpretation probability is objective, not what you think, and probability doesn't belong to anyone.

\medskip

\sml{}

\begin{verbatim}
Wins when staying with original door = 0.3359
Wins when changing door              = 0.6641
\end{verbatim}
