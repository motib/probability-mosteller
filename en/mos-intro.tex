% !TeX root = mos-en.tex

%%%%%%%%%%%%%%%%%%%%%%%%%%%%%%%%%%%%%%%%%%%%%%%%%%%%%%%%%%%%%%%%

\thispagestyle{empty}

\begin{center}
\textbf{\LARGE Mosteller's Challenging Problems in Probability}

\bigskip
\bigskip
\bigskip

\textbf{\Large Moti Ben-Ari}

\bigskip

\url{http://www.weizmann.ac.il/sci-tea/benari/}

\bigskip
\bigskip
\bigskip

Version $0.3$

\bigskip

\today

\end{center}

\vfill

\begin{center}
\copyright{} Moti Ben-Ari $2022$
 \end{center}
 
\begin{small}
This work is licensed under Attribution-ShareAlike 4.0 International. To view a copy of this license, visit \url{http://creativecommons.org/licenses/by-sa/4.0/}.

%\begin{center}\bf You are free to\end{center}
%
%\textit{Share:} copy and redistribute the material in any medium or format.
%
%\textit{Adapt:} remix, transform, and build upon the material
%for any purpose, even commercially.
%
%The licensor cannot revoke these freedoms as long as you follow the license terms.
%
%\begin{center}\bf Under the following terms\end{center}
%
%\textit{Attribution:} You must give appropriate credit, provide a link to the license, and indicate if changes were made. You may do so in any reasonable manner, but not in any way that suggests the licensor endorses you or your use.
%
%\textit{No additional restrictions:} You may not apply legal terms or technological measures that legally restrict others from doing anything the license permits.
\end{small}
\newpage

\tableofcontents

\newpage

\begin{center}
\textbf{\LARGE Introduction}
\end{center}

\addcontentsline{toc}{section}{\large Introduction}

\bigskip

\textbf{Frederick Mosteller}

Frederick Mosteller ($1916$--$2006$) founded the Department of Statistics at Harvard University and served as its chairman from $1957$ until $1971$, retiring from the university in $2003$. Mosteller was deeply interested in statistics education and wrote pioneering textbooks including \cite{pwsa} which emphasized the probabilistic approach to statistics, and \cite{bsda} which was one of the first texts on data analysis. In an interview Mosteller described the development of his approach to statistics education \cite{gse}.

\medskip

\textbf{This document}

This document is a ``reworking'' of Mosteller's delightful book \textit{Fifty Challenging Problems in Probability with Solutions} \cite{fifty}. The problems and their solutions are presented as far as possible in a manner accessible to readers with an elementary knowledge of probability, and many of the problems are accessible to secondary-school students and teachers. The problems and solutions have been rewritten to include detailed calculations and additional explanations and diagrams. I have sometimes included additional solutions.

Many of the problems have been modified to make them accessible: they are simplified, divided into subproblems and hints are provided. As a personal preference I have rephrased the problems in a more abstract way than Mosteller does and I have not given units like inches or currencies like dollars.

The numbering and titles of the problems have been retained to facilitate comparison with Mosteller's book.

Modern scientific calculators, including applications for smartphones, can perform the computations with no difficultly, although we use Stirling's approximation in two problems so that readers can see how it is used.

Basic concepts of probability are reviewed in the final section which is based on \cite{ross}. Since students may not be familiar with random variables and expectation, those concepts are presented in more detail.

Problems that are more difficult are annotated with $D$. Even a problem not marked $D$ can be difficult so do not be discouraged if you cannot solve it. However, it is worthwhile attempting to solve all the problems because any progress you make will be encouraging.

\textbf{Acknowledgments}

David Fortus, Aaron Montgomery, Michael Woltermann

\bigskip
\bigskip

\begin{center}
\textbf{\LARGE Simulations}
\end{center}

\addcontentsline{toc}{section}{\large Simulations}

\bigskip

\emph{Monte Carlo simulations} (named after the famous casino in Monaco) written in the Python 3 programming language are provided for most problems. A computer program ``performs an experiment,'' such as ``tossing a pair of dice'' or ``flipping a coin,'' a very large number of times and computes averages which are displayed. The random number generators built into Python, \verb+random.random()+ and \verb+random.randint()+, are used to obtain random outcomes for each experiment.

The programs run each simulation $10000$ times and the results are displayed to four decimal places. A simulated result will almost certainly not be exactly the same as that obtained from computing the expectation or the probability.

Python source code is available at \url{https://github.com/motib/probability-mosteller}. The files are named \verb+N-name.py+ where \verb+N+ is the problem number and \verb+name+ is the problem title.

For each simulation two results are displayed: 
\begin{itemize}
\item The theoretical value which is either a \emph{probability} or an \emph{expectation}. In general, rather than copy the values from the text they are calculated from the formulas. 
\item The result of the simulation is either a \emph{proportion} of successes relative to the number of trials corresponding to a probability, or the \emph{average} number of successes corresponding to an expectation.
\end{itemize}
It is important to understand that ``probability'' and ``expectation'' are theoretical concepts. The \emph{law of large numbers} ensures that the outcomes of many trials are very close to the theoretical values, but they won't be exactly the same. For example, the probability of obtaining a $6$ when a fair die is thrown is $1/6\approx 0.1667$. Running a simulation for $10000$ throws resulted in a range of values: $0.1684, 0.1693, 0.1687, 0.1665, 0.1656$.

The simulation programs are very short and straightforward once a problem is understood. I suggest running the simulations for various numbers of trials and other parameters to help understand their sensitivity to these parameters.

. You can run the program many times to see how the results vary