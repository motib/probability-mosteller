% !TeX root = mos-en.tex

%%%%%%%%%%%%%%%%%%%%%%%%%%%%%%%%%%%%%%%%%%%%%%%%%%%%%%%%%%%%%%%%

\thispagestyle{empty}

\begin{center}
\textbf{\LARGE Mosteller's Challenging Problems in Probability}

\bigskip
\bigskip
\bigskip

\textbf{\Large Moti Ben-Ari}

\bigskip

\url{http://www.weizmann.ac.il/sci-tea/benari/}

\bigskip
\bigskip
\bigskip

Version $0.6$

\bigskip

\today

\end{center}

\vfill

\begin{center}
\copyright{} Moti Ben-Ari $2022$
 \end{center}
 
\begin{small}
This work is licensed under Attribution-ShareAlike 4.0 International. To view a copy of this license, visit \url{http://creativecommons.org/licenses/by-sa/4.0/}.
\end{small}
\newpage

\tableofcontents

\newpage

\begin{NoHyper}
\begin{center}
\textbf{\LARGE Introduction}
\end{center}

\addcontentsline{toc}{section}{\large Introduction}

\bigskip

\textbf{Frederick Mosteller}

Frederick Mosteller ($1916$--$2006$) founded the Department of Statistics at Harvard University and served as its chairman from $1957$ until $1971$, retiring from the university in $2003$. Mosteller was deeply interested in statistics education and wrote pioneering textbooks including \cite{pwsa} which emphasized the probabilistic approach to statistics, and \cite{bsda} which was one of the first texts on data analysis. In an interview Mosteller described the development of his approach to statistics education \cite{gse}.

\textbf{This document}

This document is a ``reworking'' of Mosteller's delightful book \textit{Fifty Challenging Problems in Probability with Solutions} \cite{fifty}. The problems and their solutions are presented as far as possible in a manner accessible to readers with an elementary knowledge of probability, and many of the problems are accessible to secondary-school students and teachers. The problems and solutions have been rewritten to include detailed calculations and additional explanations and diagrams. I have sometimes included additional solutions.

To make the problems more accessible they were simplified, divided into subproblems and hints were provided. As a personal preference I have rephrased the problems in a more abstract way than Mosteller does and I have not given units like inches or currencies like dollars. The numbering and titles of the problems have been retained to facilitate comparison with Mosteller's book.

Modern scientific calculators, including applications for smartphones, can perform the computations with no difficultly, although we use Stirling's approximation in some problems so that readers can see how it is used.

Basic concepts of probability are reviewed in the final section which is based on \cite{ross}. Since students may not be familiar with random variables and expectation, those concepts are presented in more detail.

Problems that are more difficult are annotated with $D$. Even a problem not marked $D$ can be difficult so do not be discouraged if you cannot solve it. However, it is worthwhile attempting to solve all the problems because any progress you make will be encouraging.

\end{NoHyper}

\textbf{Source code}

The CC-BY-SA copyright license allows readers to freely distribute and modify the document as described in the license. \LaTeX{} and Python source code is available at:
\vspace*{-2ex}
\begin{center}
\url{https://github.com/motib/probability-mosteller}
\end{center}
\textbf{Acknowledgments}

I am indebted to Michael Woltermann for his comprehensive suggestions! Aaron M. Montgomery's help was critical to my understanding of random walks. David Fortus provided me with useful comments.

\newpage

\begin{center}
\textbf{\LARGE Simulations}
\end{center}

\addcontentsline{toc}{section}{\large Simulations}

\bigskip

\emph{Monte Carlo simulations} (named after the famous casino in Monaco) written in the Python 3 programming language are provided for most problems. A computer program performs a trial---``tossing a pair of dice'' or ``flipping a coin''---a very large number of times and computes the average or the proportion of successes. The random number generators built into Python (\verb+random.random()+, \verb+random.randint()+) were used to obtain random outcomes for each trial.

The programs run each simulation $10000$ times and the results are displayed to four decimal places. A simulated result will almost certainly not be exactly the same as that obtained from computing the expectation or the probability.

The files are named \verb+N-name.py+ where \verb+N+ is the problem number and \verb+name+ is the problem title.

For each simulation two results are displayed: 
\begin{itemize}
\item The theoretical value which is either a \emph{probability} or an \emph{expectation}. In general, rather than copy these values from the text they are calculated from the formulas. 
\item The result of the simulation is either a \emph{proportion} of successes relative to the number of trials which corresponds to a theorectical probability, or the \emph{average} number of successes which corresponds to a theoretical expectation.
\end{itemize}
It is important to understand that ``probability'' and ``expectation'' are theoretical concepts. The \emph{law of large numbers} ensures that the outcomes of many trials are very close to the theoretical values but they won't be exactly the same. For example, the probability of obtaining a $6$ when a fair die is thrown is $1/6\approx 0.1667$. Running a simulation for $10000$ throws resulted in a range of values: $0.1684, 0.1693, 0.1687, 0.1665, 0.1656$.

The simulation programs are very short and straightforward once a problem is understood. I suggest running the simulations for various numbers of trials and other parameters to help understand their sensitivity to these parameters.

For Problem 36, \textit{the gambler's ruin}, I have written a Python program that plots the results of each step. You can use this as a template to write other plotting programs.

Directory \verb+excel+ contains Excel simulations of the problems.\footnote{The simulations were contributed by Michael Woltermann (\url{mwoltermann@washjeff.edu}).}
