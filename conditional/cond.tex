
\documentclass[12pt,a4paper,leqno]{article}

\usepackage{url}

% Polyglossia Hebrew (main) and English (other)
% Fonts: David for Hebrew, Palatino/Courier for English
\usepackage{polyglossia}  
\setmainlanguage{hebrew}
\setotherlanguage{english}
\newfontfamily{\hebrewfont}{David}[Script=Hebrew]
\newfontfamily{\englishfont}{Palatino Linotype}
\newfontfamily{\englishfonttt}{Courier New}
\newfontfamily{\hebrewfonttt}{Courier New}

% Abbreviations for backwards compatibility with babel
\renewcommand*{\L}[1]{\textenglish{#1}}
\newcommand*{\R}[1]{\texthebrew{#1}}

% Displaystyle for fractions
\newcommand*{\disfrac}[2]{\displaystyle\frac{#1}{#2}}

% Layout
\textwidth=155mm
\textheight=230mm
\topmargin=0pt
\headheight=0pt
\oddsidemargin=0mm
\evensidemargin=0mm
\headsep=0pt
\parindent=0pt
\renewcommand{\baselinestretch}{1.1}
\setlength{\parskip}{0.3\baselineskip plus 1pt minus 1pt}

\begin{document}

\begin{center}
\Large
המדרון החלקלק של הסתברות מותנית

\bigskip

\Large
מוטי בן-ארי
\end{center}

כאשר סיפרתי לפרופ' אברהם הרכבי שהתחלתי להתעניין בהסתברות, הוא הזהיר אותי שמדובר בנושא "חלקלק" ואכן צדק. ומבין כל הנושאים בהסתברות הנושא החלקלק ביותר הוא הסתברות מותנית.

מאמר זה מציג שתי בעיות מפורסמות: בעיית האסיר
$[2,4]$
ובעיית מונטי הול
$[3]$.
הבעיות הללו לא קשות במיוחד אם מנתחים נכון את ההסתברות המותנית, אבל רבים וטובים טועים ואפילו מגינים בלהט על הטעויות שלהם. 

תחילה אציג את בעיית האסיר וחישוב נכון של ההסתברות שמבוססת על ספירה זהירה של המאורעות. אחר כך אסביר את הפתרון בדרך אחרת כדי להבהיר מאיפה בא החישוב המוטעה של ההסתברות המותנית. הפתרון הבא הוא פתרון אינטואיטיבי שניתן להשתמש בו כדי הבין בקלות את הפתרון הנכון לבעיית מונטי הול. המאמר מסתיים בסימולציה במחשב שהיא דרך טובה להשתכנע בנכונות הפתרון.

{\bigskip\Large
בעיית האסיר
}
\begin{quote}
שלושה אסירים 
$A,B,C$
מועמדים לשחרור מוקדם. וועדת השחרורים תשחרר שניים מהם כך שהאפשריות הן
$\{A,B\}, \{A,C\}, \{B,C\}$
בהסתברות שווה של
$1/3$
לכל זוג. לכן ההסתברות ש-%
$A$
ישוחרר היא
$2/3$.
מפקד הכלא מוסר ל-% 
$A$
מידע נכון: את זהותו האסיר האחר שישוחרר. אם ל-%
$A$
נמסר ש-%
$B$
ישוחרר, מה ההסתברות שהוא ישוחרר גם כן?
\end{quote}

הערה: יש כאן בעיתיות מסויימת. מרגע שהוחלט על זהות האסירים שישוחררו אין כאן הסתברות, כי הסתברות היא חישוב היחס הצפוי בין מספר הפעמים שמאורע יתקיים לבין מספר הניסויים שיבוצעו. מפקד הכלא יודע בוודאות את זהות האסירים. רצוי לדבר על ההסתברות מנקודת מבטו של 
$A$ 
שעדיין לא יודע מה הוחלט.


{\bigskip\Large
פתרון על ידי חישוב ההסתברות}


ארבעת המאורעות האפשריים הם:
\begin{description}
\item[$e_1$:] 
ל-%
$A$
נמסר ש-%
$B$
ישוחרר ו-%
$\{A,B\}$
ישוחררו.
\item[$e_2$:]
ל-%
$A$
נמסר ש-%
$C$
ישוחרר ו-%
$\{A,C\}$
ישוחררו.
\item[$e_3$:]
ל-%
$A$
נמסר ש-%
$B$
ישוחרר ו-%
$\{B,C\}$
ישוחררו.
\item[$e_4$:]
ל-%
$A$
נמסר ש-%
$C$
ישוחרר ו-%
$\{B,C\}$
ישוחררו.
\end{description}
ההסתברות של כל זוג להשתחרר שווה ולכן:
\[
P(e_1)=P(e_2)=P(e_3\cup e_4)=\frac{1}{3}\,.
\]
אם 
$\{B,C\}$
ישוחררו, יימסר ל-%
$A$
מידע נכון ש-%
$B$
או 
$C$
ישוחרר ובהסתברות שווה, ולכן
$P(e_3)=P(e_4)=1/6$.
אנו מעוניינים לחשב את ההסתברות ש-%
$A$
ישוחרר (מאורע
$e_1$)
בהינתן שמוסרים ל-%
$A$
ש-%
$B$
ישוחרר (מאורע
$e_1\cup e_3$):
\[
P(e_1|e_1\cup e_3) = \frac{P(e_1\cap(e_1\cup e_3))}{P(e_1\cup e_3)}=\frac{P(e_1)}{P(e_1\cup e_3)}=\frac{1/3}{1/3+1/6}=\frac{2}{3}\,.
\]
פתרון זה מדגיש את החשיבות של זיהוי מדוייק של המאורעות האפשריים בבעיה.

{\bigskip\Large
הצגה אחרת של הפתרון}


יהי 
$P(A)= P(B)=P(C)=2/3$
ההסתברויות ש-%
$A,B,C$
ישוחררו.
$A$
מעוניין בהסתברות המותנית 
$P(A|B)$
שהוא ישוחרר אם 
$B$
ישוחרר:
\[
P(A|B) = \frac{P(A\cap B)}{P(B)} = \frac{1/3}{2/3}=\frac{1}{2}\,.
\]
אבל זאת 
\textbf{לא}
ההסתברות המותנית הנכונה. המידע החדש הוא לא ש-%
$B$
ישוחרר אלא
\textbf{שנמסר}
ל-%
$A$
ש-%
$B$
ישוחרר. ההסתברות המותנת הנכונה היא
$P(A|R_{AB})$
כאשר
$R_{AB}$
הוא המאורע של-%
$A$
נמסר ש-%
$B$
ישוחרר. כדי לחשב את
$P(A|R_{AB})$:
\[
P(A|R_{AB}) = \frac{P(A\cap R_{AB})}{P(R_{AB})}
\]
עלינו לחשב את
$P(A\cap R_{AB})$
ו-%
$P(R_{AB})$.

המידע על שחרורו של
$B$
הוא אמת ולכן:
\[
P(A\cap R_{AB})=P(\{A,B\})=\disfrac{1}{3}\,.
\]
אם 
$\{B,C\}$
ישוחררו אנו מניחים שההסתברות היא
$1/2$
שמפקד יספר ל-%
$A$
ש-%
$B$
ישוחרר וגם ההסתברות היא
$1/2$
שהמפקד יספר ש-%
$C$
ישוחרר. לכן:
\begin{eqnarray*}
P(R_{AB})&=&P(\{A,B\})+\disfrac{1}{2}\cdot P(\{B,C\})=\disfrac{1}{3}+\disfrac{1}{2}\cdot \disfrac{1}{3}=\disfrac{1}{2}\\
P(A|R_{AB}) &=& \frac{P(A\cap R_{AB})}{P(R_{AB})} = \disfrac{1/3}{1/2}=\disfrac{2}{3}\,.
\end{eqnarray*}
פתרון זה מדגיש את החשיבות של זיהוי נכון של המאורע שמביא את המידע החדש עבור חישוב ההסתברות המותנית.

{\bigskip\Large
פתרון אינטואיטיבי}

בבעיית האסיר המידע ש-%
$B$
ישוחרר נמסר ל-%
$A$
\textbf{לאחר}
שהוחלט על זהות המשוחררים. מרגע זה כלל לא משנה מה נמסר לו: אפשר
לשקר לו או לא למסור לו כלום והמידע לא משנה את ההסתברות שהוא ישוחרר.

{\bigskip\Large
הבעיה של מונטי הול 
\L{(Monty Hall Problem)}}

הבעיה של מונטי הול%
\footnote{%
\selectlanguage{hebrew}%
\L{(1921-2017) Monty Hall}
נולד בשם 
\L{Monte Halparin}
למשפחת יהודית בויניפג, קנדה.}
ידועה לשימצה. לא אכתוב על ההיסטוריה ועל ההיסטריה של הבעיה. מי שמעוניין ימצא מידע רב בויקיפדיה
$[3]$.
נהרות של דיו נשפכו על פרשנויות אפשרויות לההצגה המקורית של הבעיה. כדי למנוע בלבול אני אפרט לפרטי פרטים את כללי המשחק ואת ההנחות שלעתים נשארו סמויות:
\begin{enumerate}
\item
בשעשועון בטלוויזיה המתחרה ניצב לפני שלוש דלתות. מאחרי כל אחת מהדלתות נמצא אחד הפרסים שהם מכונית אחת ושתי עזים.
\item
המתחרה מעדיף לזכות במכונית ולא בעז!%
\footnote{%
לא מצאתי הנחה סמויה זו בספרות אבל אם רוצים לדייק כדאי להציף אותה.}
\item
מיקום כל פרס הוא אקראי בהתפלגות אחידה (סיכויים שווים).
\item
המנחה יודע את המיקום של כל אחד מהפרסים. 
\item
המתחרה חייב לבחור דלת אחת.
\item
לפני שמגלים למתחרה איזה פרס נמצא מאחורי הדלת שבחר, המנחה חייב לפתוח את אחת הדלתות שמאחוריהן נמצאות העזים. קיימות שתי אפשרויות:
\begin{itemize}
\item
אם המתחרה בחר בדלת עם המכונית, הבחירה של המנחה בין שתי הדלתות האחרות היא אקראית בהתפלגות אחידה.
\item
אם המתחרה בחר בדלת עם עז, המנחה חייב לפתוח את הדלת עם העז השניה.
\end{itemize}
\item
לאחר פתיחת הדלת, המתחרה חייב להחליט אם להישאר עם הבחירתו המקורית או לשנות אותה ולבחור בדלת השניה שלא נפתחה.
\item
המנחה פותח את הדלת שנבחרה והמתחרה זוכה בפרס שנמצא שם.
\end{enumerate}
\begin{quote}
מה האסטרטגיה העדיפה עבור המתחרה: להישאר עם הבחירה המקורית או לשנות אותה?
\end{quote}
לפי פתרון שכיח ההסתברות שהמכונית נמצאת מאחורי הדלת המקורית משתנה מ-%
$1/3$
ל-%
$1/2$
כי המכונית נמצאת מאחורי אחת משתי הדלתות שעדיין סגורות, ולכן לא משנה אם המתחרה בוחר להישאר עם החלטתו או לשנותה. פתרון זה 
\textbf{שגוי}
כי אין כאן שני ניסויים בלתי-תלויים. לאחר פתיחת הדלת, המנחה
\textbf{לא}
מטיל מטבע כדי להחליט מחדש איפה לשים את המכונית ואיפה לשים את העז.

{\bigskip\Large 
מבעיית האסיר לבעיית מונטי הול}

נניח ששינו את כללי המשחק כך שהמתחרה
\textbf{לא}
רשאי לשנות את החלטתו! ההסתברויות לא משתנות: ההסתברות לזכות במכונית נשארת
$1/3$
והסתברות לזכות בעז נשארת
$2/3$.
מצב זה שקול לבעיית האסיר כי האסיר 
$A$
לא רשאי לשנות את ההחלטה מי ישוחרר ולכן לא משנה מה אומרים לו. עם הכללים החדשים לא משנה איזו דלת המנחה פותח, המתחרה לא יכול לעשות כלום.

בעיית מונטי הול שונה. הכלל שהמתחרה רשאי לשנות את החלטתו היא הסיבה היחידה שיש משמעות לפתיחת הדלת. שוב, ההסתברויות לא משתנות: הההסתברות שהמכונית נמצאת האחורי הדלת שנבחרה בהחלטה המקורית היא
$1/3$,
והסתברות שהיא נמצאת האחורי שתי הדלתות האחרות היא
$2/3$.
אבל כעת הסתברות
$2/3$
"מרוכזת" בדלת אחת, זו שהמנחה לא פתח, ולכן, המתחרה יכול להכפיל את סיכוייו לזכות מ-%
$1/3$
ל-%
$2/3$
על ידי שינוי ההחלטה.

{\bigskip\Large
סימולציה}

רבים, כולל מתמטיקאים, משתכנעים בנכונות הפתרון רק כאשר רואים  תוצאות של סימולציה. כתבתי תכנית פייתון של מספר שורות המבצעת סימולציה עבור מספר משחקים שאפשר לבחור. התכנית משתמשת במחולל מספרים 
\textbf{פסאודו-אקראיים}
שהם סדרה של מספרים שאינם אקראיים, אבל אם מחשבים את ההתפלגותם אין הבדל ממשעותי בינם לבין התפלגות של מספר אקראיים. בתכנית הסימולציה של בעיית מונטי הול הדפסתי את ההתפלגויות של מיקום המכונית כדי שנראה שאפשר לסמוך על מחולל המספרים הפסאודו-אקראיים:
\[
\begin{array}{r|r|r|r}
3\;\textrm{דלת}&
2\;\textrm{דלת}&
1\;\textrm{דלת}&
\textrm{משחקים}
\\\hline
0.332 & 0.330 & 0.338 & 1000\\
0.342 & 0.336 & 0.322 & 10000\\
0.333 & 0.334 & 0.332 & 100000
\end{array}
\]
הנה היחס בין מספר הנצחונות למספר המשחקים עבור שתי האסטרטגיות: להישאר עם הדלת המקורית או לשנות לדלת השניה:
\[
\begin{array}{r|r|r}
\textrm{לשנות} &
\textrm{להישאר} &
\textrm{משחקים}
\\\hline
0.670 & 0.330 & 1000\\
0.656 & 0.344 & 10000\\
0.665 & 0.335 & 100000
\end{array}
\]
השתכנעתם?

{\bigskip\Large
תכנית בפייתון}

\selectlanguage{english}
\begin{verbatim}
import random
games = 1000
stay  = change = 0

# doors: False for goat, True for car
for g in range(games):
    doors = [False, False, False]
    doors[random.randint(0,2)] = True
    # Wlog car is behind door 0
    if doors[0]:     stay += 1
    if not doors[0]: change += 1

print('stay = {:.3f}, change = {:.3f}'.
      format(stay/games, change/games))
\end{verbatim}
\selectlanguage{hebrew}

{\bigskip\Large
סיכום}

קשה להפריז בעניין ששתי הבעיות הללו עוררו. לכל אחד מהערכים בויקיפדיה 
$[3,4]$
מצורפת רשימה של עשרות מקורות. חילוקי הדעות התעוררו בגלל אי-הקפדה על ניסוח מדויק של הבעיה כולל הצפת כל ההנחות הסמויות, ולכן טרחתי לנסח אותה לפרטי פרעטים. הקשיים בפתרון נובעים מחוסר זהירות בזיהוי המאורעות השונים ובניסוח לקוי של ההסתברות המותנית. 

הבעיות הללו מעניינות ואף משעשעות והתמודדות איתן עשויה לתרום תרומה של ממש להבנת המושג החלקלק הסתברות מותנית.

{\bigskip\Large
תודה}

אני מודה לאביטל אלבוים-כהן ולדיויד פורטס עבור הערותיהם שתרמו רבות לשיפור המאמר.

{\bigskip\Large
מקורות}

מצאתי מעט מקורות בעברית והם לא מעמיקים במיוחד 
$[5,6,7]$.
אני ממליץ על הערכים הנרחבים בויקיפדיה באנגלית. אזהרה: 
\L{Mosteller}
$[2]$
משתמש בשם
\L{the prisoner's dilemma}
אולם שם זה משמש בדרך כלל לבעיה מפורסמת בתורת המשחקים. למעשה אין כאן "דילמה" אלא שאלה על ההסתברות של מאורע. בויקיפדיה בעיית האסיר ידועה כ-%
\L{three prisoners problem}.
בעברית אין כלל ערך על בעיית האסירים והיא מוזכרת רק במספר משפטים בערך של בעיית מונטי הול. השתמשתי ב-"בעיית האסיר" כדי לבדל אותה מדילמת האסיר. הסבר מתמטי מוקפד של שתי הבעיות נמצא ב-%
$[1]$.

\selectlanguage{english}

[1] Matthew Carlton. Pedigrees, prizes, and prisoners: The misuse of conditional probability. \textit{Journal of Statistics Education} (2)13, 2005, \url{https://doi.org/10.1080/10691898.2005.11910554}.

[2] Frederick Mosteller. \textit{Fifty Challenging Problems in Probability with Solutions}. Dover, 1965.

[3] Wikipedia. Monty Hall problem.

[4] Wikipedia. Three Prisoners problem.

[5]
\R{%
גדי אלכסנדרוביץ’. הבעיה של מונטי הול. 2008.}\\
\url{https://gadial.net/2008/02/23/monty_hall/}.

[6]
\R{וקיפדיה. בעיית מונטי הול}.

[7]
\R{%
אלכס קופרמן. הסתברות מותנית כמקור לפרדוקסים ותוצאות מפתיעות. על"ה 22, 1998.}

\bigskip

\selectlanguage{hebrew}
\begin{center}
פרופ' (אמריטוס) מוטי בן-ארי\\
המחלקה להוראת המדעים\\
מכון ויצמן למדע
\end{center} 

\end{document}
