% !TeX root = mos-he.tex

%%%%%%%%%%%%%%%%%%%%%%%%%%%%%%%%%%%%%%%%%%%%%%%%%%%%%%%%%%%%%

\refstepcounter{problem}  % 11. Silent cooperation

%%%%%%%%%%%%%%%%%%%%%%%%%%%%%%%%%%%%%%%%%%%%%%%%%%%%%%%%%%%%%

\refstepcounter{problem}  % 12. Quo vadis?

%%%%%%%%%%%%%%%%%%%%%%%%%%%%%%%%%%%%%%%%%%%%%%%%%%%%%%%%%%%%%

\begin{prob}{דילמת האסיר}{D}{The prisoner's dilemma}

שלושה אסירים 
$A,B,C$
מועמדים לשחרור מוקדם. וועדת השחרורים ישחרר שניים מהם עם הסתברות שווה ל-%
$\{A,B\}, \{A,C\}, \{B,C\}$, 
כך שהסיכוי ש-%
$A$
ישוחרר הוא
$2/3$.
לאסיר 
$A$
נמסר שמו של אחד האסירים האחרים שישוחרר. אם מוסרים לו ש-%
$B$
מה הסיכוי ש-%
$A$
ישוחרר גם כן?

\L{\cite{carlton}}
הוא מאמר מקיף על דילמת האסיר ועל בעיית 
\L{Monty Hall}
הדומה.
\end{prob}

\solution{1}

יהיו 
$P(A), P(B), P(C)$
ההסתברות ש-%
$A,B,C$
ישוחררו.
$A$
מעוניין בהסתברות המותנית 
$P(A|B)$
שהוא ישוחרר אם 
$B$
ישוחרר. נדמה שהחישוב שלהלן הוא מה שאנחנו רוצים:
\[
P(A|B) = \frac{P(A\cap B)}{P(B)} = \frac{1/3}{2/3}=\frac{1}{2}\,.
\]
אבל זאת לא ההסתברות הנכונה! יהי 
$R_{AB}$
האירוע של-%
$A$
נמסר ש-%
$B$
ישוחרר. ההסתברות שיש לחשב היא
$P(A|R_{AB})$:
\[
P(A|R_{AB}) = \frac{P(A\cap R_{AB})}{P(R_{AB})}\,.
\]
אנו מניחים שהמידע על שחרורו של
$B$
הוא אמת ולכן:
\[
P(A\cap R_{AB})=P(\{A,B\})=\disfrac{1}{3}\,.
\]
כעת:
\[
P(R_{AB})=P(\{A,B\})+P(\{B,C\})=\disfrac{1}{3}+\disfrac{1}{2}\cdot \disfrac{1}{3}=\disfrac{1}{2}\,.
\]
אם התכנית היא ש-%
$\{B,C\}$
ישוחררו, ניתן למסור ל-%
$A$
או ש-%
$B$
ישוחרר או ש-%
$C$
ישוחרר ולכן הגורם 
$1/2$.
מכאן:
\[
P(A|R_{AB}) = \frac{P(A\cap R_{AB})}{P(R_{AB})} = \disfrac{1/3}{1/2}=\disfrac{2}{3}\,,
\]
כך שמסירה ל-%
$A$
ש-%
$B$
ישוחרר לא משנה את ההסתברות שהוא ישוחרר.

\solution{2}

ארבעת האירועים האפשריים הם:
\begin{description}
\item[$e_1$:] 
ל-%
$A$
נמסר ש-%
$B$
ישוחרר ו-%
$\{A,B\}$
שוחררו.
\item[$e_2$:]
ל-%
$A$
נמסר ש-%
$C$
ישוחרר ו-%
$\{A,C\}$
שוחררו.
\item[$e_3$:]
ל-%
$A$
נמסר ש-%
$B$
ישוחרר ו-%
$\{B,C\}$
שוחררו.
\item[$e_4$:]
ל-%
$C$
נמסר ש-%
$B$
ישוחרר ו-%
$\{B,C\}$
שוחררו.
\end{description}
ההסתברות של כל זוג לשחרור שווה ולכן:
\[
P(e_1)=P(e_2)=P(e_3\cup e_4)=\frac{1}{3}\,.
\]
נניח שאם 
$\{B,C\}$
ישוחררו, בהסתברות שווה ימסרו ל-%
$A$
ש-%
$B$
ישוחרר או ש-%
$C$
ישוחרר, ולכן
$P(e_3)=P(e_4)=1/6$.
מכאן שההסתברות ש-%
$A$
ישוחרר בהינתן האירוע 
$R_{AB}=e_1\cup e_3$
שנמסר לו ש-%
$B$
ישוחרר היא:
\[
P(A|R_{AB}) = \frac{P(e_1\cap(e_1\cup e_3))}{P(e_1\cup e_3)}=\frac{P(e_1)}{P(e_1\cup e_3)}=\frac{1/3}{(1/3)+(1/6)}=\frac{2}{3}\,.
\]

\solution{3}

חידה שמיוחסת ל-%
\L{Abraham Lincoln}
שואל: "אם תקרא לזנב של כלב רגל, כמה רגליים יש לכלב?" התשובה היא שלקרוא לזנב רגל לא הופך אותו לרגל ולכן לכלב עדיין יש ארבע רגליים. ברור שאם
$A$
יודע את העתיד המחכה ל-%
$B$
זה לא משנה את הסיכוי שלו לשיחרור.

%%%%%%%%%%%%%%%%%%%%%%%%%%%%%%%%%%%%%%%%%%%%%%%%%%%%%%%%%%%%%

\begin{prob}{איסוף תלושים}{S}{Collecting coupons}
נתון סדרה של קופסאות ובתוך כל אחת נמצא תלוש עם אחד המספרים 
$1$
עד
$5$.
את שולפת תלוש אחד מכל קופסה אחת אחרי השנייה.

\que{1}
מה התוחלת של מספר התלושים שיש לשלוף עד שתקבלי את כל חמשת המספרים?

\que{2}
פתחי נוסחה לתוחלת ל-%
$n$
מספרים.

\textbf{רמז:}
תשתמשי בפתרון לבעיה~4 בעמוד%
~\pageref{p.four}
והקירוב לסכום של מספרים הורמוניים 
(עמוד%
~\pageref{p.harmonic}).
\end{prob}

\solution{}

\ans{1}
מה התוחלת של מספר השליפות עד שאת מקבלת מספר
\textbf{שונה}
מכל הקודמים? לפי בעייה~4 התוחלת היא
$1/p$
כאשר 
$p$
היא ההסתברות לשליפת מספר שונה. עבור השליפה הראשונה, ההסתברות היא 
$1$
כך שהתוחלת היא גם
$1$.
עבור השליפה השנייה ההסתברות היא 
$4/5$
כך שהתוחלת היא 
$5/4$
וכך הלאה. לכן:
\[
E(\textrm{המספרים חמשת כל}) = \frac{5}{5}+\frac{5}{4} + \frac{5}{3} + \frac{5}{2} + \frac{5}{1} = \frac{}{} =\frac{1370}{120}\approx 11.4167\,.
\]
\ans{2}
נשמתמש באותה שיטה ובקירוב לסכום המספרים ההרומוניים (עמוד%
~\pageref{p.harmonic}):
\[
E(\textrm{המספרים}\;n \;\textrm{כל}) = n\left(\frac{1}{n}+\frac{1}{n-1} + \cdots \frac{1}{2} + \frac{1}{1}\right) =nH_n\approx n\left(\ln n + \frac{1}{2n} + 0.5772\right)\,. 
\]
עבור
$n=5$
מתקבל:
\[
E(\textrm{המספרים חמשת כל}) =5H_5\approx 5(\ln 5 + \frac{1}{10} + 0.5772) \approx 11.4332\,.
\]

\textbf{סימולציה}
\selectlanguage{english}
\begin{verbatim}
For  5 coupons:
Expectation of draws = 11.9332
Average draws        = 11.4272
For 10 coupons:
Expectation of draws = 29.7979
Average draws        = 29.2929
For 20 coupons:
Expectation of draws = 72.4586
Average draws        = 72.2136
\end{verbatim}
\selectlanguage{hebrew}

%%%%%%%%%%%%%%%%%%%%%%%%%%%%%%%%%%%%%%%%%%%%%%%%%%%%%%%%%%%%%

\begin{prob}{שורה בתיאטרון}{S}{The theater row}
סדר שמונה מספרים זוגיים ושבעה מספרים אי-זוגיים בשורה בצורה אקראית, למשל:
\[
10\quad 12\quad 3\quad 2\quad 9\quad 6 \quad 1\quad 13\quad 7\quad 10\quad 3\quad 8\quad 8\quad 5\quad 20\,,
\]
שנוכל לכתוב כך:
\[
E\quad E\quad O\quad E\quad O\quad E \quad O\quad O\quad O\quad E\quad O\quad E\quad E\quad O\quad E\,,
\]
כי המספרים עצמם אינם חשובים.

מה התוחלת שלמספר הזוגות השכנים שהם זוג/אי-זוגי או אי-זוגי/זוגי?

בדוגמה יש $10$ זוגות שכנים שהם
$EO$ 
או 
$OE$.

\textbf{רמז:}
התייחס לכל זוג שכנים בנפרד. מה ההסתברות שהם שונים?
\end{prob}

\solution{}

הטבלה שלהן מראה את עשרת הסידורים השונים עבור שלושה מספרים זוגיים ושני מספרים אי-זוגיים. מספר הזוגות השכנים השונים הוא 
$24$
והממוצע הוא
$24/10=2.4$.
\[
\begin{array}{c|c|@{\hspace{2pt}}|c|c}
\textrm{סידור}&\textrm{זוגות}&\textrm{סידור}&\textrm{זוגות}\\\hline
EEEOO & 1&
EEOEO & 3\\
EEOOE & 2&
EOEOE & 4\\
EOEEO & 3&
EOOEE & 1\\
OEEOE & 3&
OEEEO & 2\\
OEOEE & 3&
OOEEE & 1\\
\end{array}
\]
נחזור לדוגמה עם
$15$
מספרים. תהי
$P_d$
ההסתברות שזוג נתון בסידור הוא 
$EO$
או
$OE$.
אזי:
\[
P_d =P(EO) + P(OE) = \frac{8}{15}\cdot \frac{7}{14} + \frac{7}{15}\cdot \frac{8}{14} = 2\cdot \frac{8}{15}\cdot \frac{7}{14} = \frac{8}{15}\,.
\]
תהי
$E_d$
התוחלת של מספר הזוגות בסידור שהם
$EO$
או 
$OE$.
זוג 
$(EO,OE)$
תורם $1$ למספר הזוגות השונים וזוג
$(EE,OO)$
תורם $0$:
\[
E_d =
\sum_{\textrm{\footnotesize זוגות}} 1\cdot P_d= 14\cdot \frac{8}{15} \approx 7.4667\,.
\]
עבור עשרה מספרים:
\begin{eqnarray*}
P_d &=& P(EO) + P(OE) = \frac{3}{5}\cdot \frac{2}{4} + \frac{2}{5}\cdot \frac{3}{4} = \frac{3}{5}\\
E_d &=& 4\cdot \frac{3}{5}=\frac{12}{5}=2.4\,.
\end{eqnarray*}

\textbf{Simulation}
\selectlanguage{english}
\begin{verbatim}
For  5 places:
Expectation of different pairs = 2.4000
Average different pairs        = 2.3855
For 15 places:
Expectation of different pairs = 7.4667
Average different pairs        = 7.4566
For 27 places:
Expectation of different pairs = 13.4815
Average different pairs        = 13.4835
For 49 places:
Expectation of different pairs = 24.4898
Average different pairs        = 24.4725
\end{verbatim}
\selectlanguage{hebrew}

%%%%%%%%%%%%%%%%%%%%%%%%%%%%%%%%%%%%%%%%%%%%%%%%%%%%%%%%%%%%%

\begin{prob}{האם השני בדירוג יזכה המקום שני?}{S}{Will the second-best be runner-up?}
לשמונה שחקים בתחרות 
$\{a_1,\ldots,a_8\}$
נקבעים משחקים 
$\{g_1,\ldots,g_8\}$
בצורה אקראית כך ששחקן
$a_{k_{i}}$
משחק את המשחק הראשון שלו במקום
$g_{k_{i}}$ (\R{איור}~\ref{f.tournament}).
השחקנים מדורגים כך שהטוב ביותר הוא 
$a_1$
והגרוע ביותר הוא 
$a_8$.
השחקן הטוב יותר לעולם ינצח שחקן פחות. ברור ששחקן
$a_1$
ינצח בתחרות.

\que{1}
מה ההסתברות שהשחקן
$a_2$
יזכה במקום השני כאשר הוא משחק נגד 
$a_1$
בגמר ומפסיד?

\que{2}
עבור
$2^n$
שחקנים, מה ההסתברות שהשחקן
$a_2$
יזכה במקום השני כאשר הוא משחק נגד 
$a_1$
בגמר ומפסיד?
\end{prob}
\begin{figure}[tb]
\begin{center}
\begin{tikzpicture}[scale=.75]
\foreach \n in {1,2,3,4,5,6,7,8}
  \node (\n) at (0,\n*10mm) {$g_{\n}$};
\foreach \n in {1,2,3,4}
  \node[inner sep=-4pt] (r\n) at (30mm,-5mm+\n*20mm) {};
\foreach \n/\r in {1/1,2/1,3/2,4/2,5/3,6/3,7/4,8/4}
  \draw (\n) --
     node[fill=white] {$a_{k_{\n}}$} +(20mm,0) -- (r\r);
\foreach \n/\v in {1/25mm,2/65mm}
  \node[inner sep=-5pt] (rr\n) at (60mm,\v) {};
\foreach \n/\r in {1/1,2/1,3/2,4/2}
  \draw ($(r\n)+(-1pt,0)$) -- +(20mm,0) -- (rr\r);
\node[inner sep=-5pt] (rrr) at (90mm,45mm) {};
\foreach \n/\r in {1,2}
  \draw ($(rr\n)+(-1pt,0)$) -- +(20mm,0) -- (rrr); 
\draw ($(rrr)+(-1pt,0)$) -- +(20mm,0);
\node at (22mm,88mm) {\textrm{גמר רבע}};
\node at (52mm,88mm) {\textrm{גמר חצי}};
\node at (82mm,88mm) {\textrm{גמר}};
\end{tikzpicture}
\end{center}
\caption{טבלת משחקים לתחרות}\label{f.tournament}
\end{figure}

\solution{}

\ans{1}
אם 
$a_1$
משחק באחד משחקים
$\{g_1,g_2,g_3,g_4\}$
אף שחקן המשחקים במשחקים הללו לא יגיע לגמר, לכן 
$a_2$
חייב לשחק באחד מהשחקים
$\{g_5,g_6,g_7,g_8\}$.
המסקנה המתבקש היא שההסתרות ש-%
$a_2$
יזכה במקום השני היא
$1/2$
כי הוא חייב לשחק באחד מארבעת המשחקים
$\{g_1,g_2,g_3,g_4\}$.
אולם, 
$a_2$
חייב לשחק באחד מארבעה המשחקים מתוך שבעת המשחקים ש-%
$a_1$
לא משחק בו וההסתברות היא
$4/7$.

\ans{2}
באופן דומה, מתוך
$2^n-1$
המשחקים בהם 
$a_1$
לא משחק, 
$a_2$
חייב לשחק באחד מ-%
$2^{n-1}$
המשחקים במחצית הטבלה שלא כוללת את
$a_1$.
מכאן:
\[
P(\textrm{בגמר השני נגד אחד משחקים} \; a_1,a_2)=\frac{2^{n-1}}{2^n-1}\,.
\]

\textbf{סימולציה}
\selectlanguage{english}
\begin{verbatim}
For   8 players:
Probability a2 is runner-up                = 0.5714
Proportion of games where a2 is runner-up  = 0.5707
For  32 players:
Probability a2 is runner-up                = 0.5161
Proportion of games where a2 is runner-up  = 0.5184
For 128 players:
Probability a2 is runner-up                = 0.5039
Proportion of games where a2 is runner-up  = 0.5060
\end{verbatim}
\selectlanguage{hebrew}

%%%%%%%%%%%%%%%%%%%%%%%%%%%%%%%%%%%%%%%%%%%%%%%%%%%%%%%%%%%%%

\begin{prob}{זוג אבירים}{D,S}{Twin knights}

לשמונה שחקים בתחרות 
$\{a_1,\ldots,a_8\}$
נקבעים משחקים 
$\{g_1,\ldots,g_8\}$
בצורה אקראית כך ששחקן
$a_{k_{i}}$
משחק את המשחק הראשון שלו במקום
$g_{k_{i}}$ (\R{איור}~\ref{f.tournament}).
לכל
$i,j$, 
ההסתברות ש-%
$a_i$
מנצחת במשחק נגד 
$a_j$
היא 
$1/2$
כמו גם ההסתברות ש-%
$a_j$
מנצחת את
$a_i$.

\que{1} 
מה ההסתברות שהשחקנים
$a_1,a_2$
משחקים משחק אחת נגד השנייה.

\que{2}
עבור
$2^n$ 
שחקנים, מה ההסתברות שהשחקנים
$a_1,a_2$
משחקים משחק אחת נגד השנייה.
\end{prob}

\solution{}

\ans{1} 
ללא הגבלת הכלליות נקבע ש-%
$a_1$
משחקת במשחק
$g_1$.
מה האפשרויות בהן 
$a_1,a_2$
ישחקו אחת נגד השנייה. בהסתברות 
$1/7$
נקבע ש-%
$a_2$
משחקת במשחק
$g_2$.
בהסתברות
$2/7$
נקבע ש-%
$a_2$
משחקת במשחק 
$g_3$
או
$g_4$,
אבל היא לא משחקת נגד 
$a_1$
אלא אם
\textbf{שתיהן}
מנצחות במשחק הראשון שלהן, כך שיש להכפיל את ההסתברות ב-%
$1/4$.
בהסתברות
$4/7$
נקבע ש-%
$a_2$
משחקת במשחק
$g_5,g_6,g_7,g_8$, 
אבל היא לא משחקת נגד 
$a_1$
\textbf{שתיהן}
מנצחות בשני המשחקים שלהן, כך שיש להכפיל את ההסתברות ב-%
$1/16$.
מכאן:
\[
P(a_1, a_2\;\textrm{play each other})=\frac{1}{7} + \frac{1}{4}\cdot \frac{2}{7} + \frac{1}{16}\cdot \frac{4}{7} =\frac{1}{4}\,.
\]

\ans{2}
תהי 
$P_n$
ההסתברות שבתחרות עם 
$2^n$
שחקנים,
$a_1$
ו-%
$a_2$
משחקות אחת נגד השנייה. ראינו ש-%
$P_3=1/4$.
מה עם
$P_4$?
באותה שיטה:
\begin{eqnarray*}
P_4 &=& \frac{1}{15} + \frac{1}{4}\cdot \frac{2}{15}  + \frac{1}{16}\cdot \frac{4}{15}  + \frac{1}{64}\cdot \frac{8}{15} \\
&=&\frac{1}{15}\left(1+\frac{1}{2}+\frac{1}{4}+\frac{1}{8}\right)=\frac{1}{8}\,.
\end{eqnarray*}
השערה סבירה היא ש-%
$P_n=1/2^{n-1}$.

\textbf{Proof 1:}
באותה שיטה תוך שימוש בנוסחה לסכום של סדרה הנדסית:
\begin{eqnarray*}
P_n&=&\frac{1}{2^n-1}\sum_{i=0}^{n-1}2^i\cdot \left(\frac{1}{2}\right)^{2i}\\
&=&\frac{1}{2^n-1}\sum_{i=0}^{n-1}2^{-i}\\
&=&\frac{1}{2^n-1}
  \left(
    \frac{1-\left(\frac{1}{2}\right)^{(n-1)+1}}
         {1-\frac{1}{2}}
  \right)=\frac{1}{2^{n-1}}\,.
\end{eqnarray*}

\textbf{Proof 2:}
באינדוקציה. טענת הבסיס היא:
$P_3=1/4=1/2^{3-1}$.

יש שני צעדי אינדוקציה:

\textbf{מקרה 1:} 
נקבע ש-%
$a_1$
ו-%
$a_2$
משחקות בחצאים שונים של התחרות:
\[
P(\textrm{שונים בחצאים משחקות}\;a_1,a_2)=\frac{2^{n-1}}{2^n-1}\,.
\]
הן יכולות להיפגש רק במשחק הגמר ולכן כל אחת חייבת לנצח בכל 
$n-1$
המשחקים שלהן:
\begin{equation}\label{eq.17a}
P(\textrm{שונים בחצאים משחקות אם נפגשות}\; a_1,a_2)=\frac{2^{n-1}}{2^n-1} \left(\frac{1}{2}\right)^{n-1} \left(\frac{1}{2}\right)^{n-1}=\frac{2^{-(n-1)}}{2^n-1}\,.
\end{equation}
\textbf{מקרה 2:}
נקבע ש-%
$a_1$
ו-%
$a_2$
משחקות באותו חצי תחרות:
\[
P(a_1,a_2\;\textrm{שונים בחצאים משחקות})=\frac{2^{n-1}-1}{2^n-1}\,.
\]
בגלל ששתי השחקניות משחקות באותו חצי הבעיה מצטמטת בדרישה למצוא
$P_{n-1}$.
לפי הנחת האינדוקציה:
\begin{equation}\label{eq.17b}
P(\textrm{חצי באותו משחקות אם נפגשות}\; a_1,a_2)=\frac{2^{n-1}-1}{2^n-1}\cdot \frac{1}{2^{n-2}}=\frac{2^{1}-2^{-(n-2)}}{2^n-1}\,.
\end{equation}
ממשוואות%
~\ref{eq.17a}, \ref{eq.17b}
נקבל:
\[
\renewcommand*{\arraystretch}{2.2}
\begin{array}{rcl}
P_n&=&\disfrac{2^{-(n-1)}}{2^n-1}+\disfrac{2^{1}-2^{-(n-2)}}{2^n-1}\\
&=&\disfrac{2^{n-1}}
        {2^{n-1}}\cdot 
   \disfrac{2^{-(n-1)}+2^{1}-2^{-(n-2)}}
        {2^n-1}\\
&=&\disfrac{1}
        {2^{n-1}}\cdot 
   \disfrac{2^0+2^n-2^1}
        {2^n-1}=\disfrac{1}{2^{n-1}}\,.
\end{array}
\]

\textbf{סימולציה}
\selectlanguage{english}
\begin{verbatim}
For   8 players:
Probability that a1, a2 meet = 0.2500
Proportion a1, a2 meet       = 0.2475
For  32 players:
Probability that a1, a2 meet = 0.0625
Proportion a1, a2 meet       = 0.0644
For 128 players:
Probability that a1, a2 meet = 0.0156
Proportion a1, a2 meet       = 0.0137
\end{verbatim}
\selectlanguage{hebrew}

%%%%%%%%%%%%%%%%%%%%%%%%%%%%%%%%%%%%%%%%%%%%%%%%%%%%%%%%%%%%%

\begin{prob}{תוצאה שווה בהטלת מטבע}{S}{An even split at coin tossing}
\que{1}
אם אתה זורק מטבע הגון 
$20$
פעמים, מה ההסתברות לקבל "עץ" בדיוק 
$10$
פעמים?

\que{2}
אם אתה זורק מטבע הגון 
$40$
פעמים, מה ההסתברות לקבל "עץ" בדיוק 
$20$
פעמים?
\end{prob}

\solution{}

\ans{1}
המטבע הוגן ולכן ההסתברות מתקבלת מהמקדם הבינומי:
\[
P(\textrm{הטלות}\; 20\textrm{ב- "עץ"}\;10)=
{20 \choose 10} \left(\frac{1}{2}\right)^{10}\left(\frac{1}{2}\right)^{10}\approx 0.1762\,.
\]

\ans{2} 
אפשר לשער שהתוצאה תהיה זהה לשאלה הקודמת אבל:
\[
P(\textrm{הטלות}\;40\textrm{ב- "עץ"}\;20)=
{40 \choose 20} \left(\frac{1}{2}\right)^{20}\left(\frac{1}{2}\right)^{20}\approx 0.1254\,.
\]
לפי חוק המספרים הגדולים מספר "עץ" ומספר "לפי" יהיו "בערך" שווים
\cite[Section~8.4]{ross},
אבל הסיכוי נמוך מאוד שיהיו שווים בדיוק, והסיכוי לאירוע זה הופך להיות קטן יותר ככל שמספר ההטלות גדל.

\textbf{סימולציה}
\selectlanguage{english}
\begin{verbatim}
Probability of 10 heads for  20 tosses = 0.1762
Proportion  of 10 heads for  20 tosses = 0.1790
Probability of 20 heads for  40 tosses = 0.1254
Proportion  of 20 heads for  40 tosses = 0.1212
Probability of 50 heads for 100 tosses = 0.0796
Proportion  of 50 heads for 100 tosses = 0.0785
\end{verbatim}
\selectlanguage{hebrew}

%%%%%%%%%%%%%%%%%%%%%%%%%%%%%%%%%%%%%%%%%%%%%%%%%%%%%%%%%%%%%

\begin{prob}{\L{\small Isaac Newton} עוזר ל-\L{\small Samuel Pepys}}{S}{Isaac Newton helps Samuel Pepys}
\que{1} 
מה ההסתברות לקבל
\textbf{לפחות}
$6$
\textbf{אחד}
כאשר מטילים 
$6$
קוביות.

\que{2} 
מה ההסתברות לקבל
\textbf{לפחות שני}
$6$
כאשר מטילים 
$12$
קוביות.

\que{3}
פתח נוסחה להסתברות לקבל לפחות
$n$ $6$
כאשר מטילים 
$6n$
קוביות.
\end{prob}

\solution{}

\ans{1} 
ההסתברות היא המשלים להסתברות לקבל אפס 
$6$
ב-%
$6$
הטלות, שהיא המכפלה של לקבל מספר שונה מ-%
$6$
בכל ההטלות:
\[
P(\textrm{אחד} \; 6\; \textrm{לפחות})=1-\left(\frac{5}{6}\right)^6\approx 0.6651\,.
\]
\ans{2}
ההסתברות היא המשלים להסתברות לקבל אפס או אחד
$6$
ב-%
$12$
הטלות:
\[
P(6\; \textrm{לפחות שני})=1-\left(\frac{5}{6}\right)^{12}-{12\choose 1}\left(\frac{1}{6}\right)^{1}\left(\frac{5}{6}\right)^{11}\approx 0.6187\,.
\]
לאירוע זה הסתברות קטנה יותר מהאירוע הקודם.

\ans{3} 
ההסתברות היא המשלים להסתברות לקבל פחות מ-%
$n$ $6$
ב-%
$6n$
הטלות:
\begin{eqnarray*}
P(6\; n\;\textrm{לפחות})&=&
  1-{6n \choose 0}\left(\frac{1}{6}\right)^0\left(\frac{5}{6}\right)^{6n-0}-
  {6n\choose 1}\left(\frac{1}{6}\right)^{1}\left(\frac{5}{6}\right)^{6n-1}-\cdots\\
&=&1-\sum_{i=0}^{n-1}{6n\choose i}\left(\frac{1}{6}\right)^{i}\left(\frac{5}{6}\right)^{6n-i}\,.
\end{eqnarray*}

\textbf{סימולציה}
\selectlanguage{english}
\begin{verbatim}
For   6 dice to throw  1 sixes:
Probability = 0.6651
Proportion  = 0.6566
For  12 dice to throw  2 sixes:
Probability = 0.6187
Proportion  = 0.6220
For  18 dice to throw  3 sixes:
Probability = 0.5973
Proportion  = 0.5949
For  96 dice to throw 16 sixes:
Probability = 0.5424
Proportion  = 0.5425
For 360 dice to throw 60 sixes:
Probability = 0.5219
Proportion  = 0.5250
\end{verbatim}
\selectlanguage{hebrew}

%%%%%%%%%%%%%%%%%%%%%%%%%%%%%%%%%%%%%%%%%%%%%%%%%%%%%%%%%%%%%


\begin{prob}{דו-קרב משולש}{S}{(The three-cornered duel)}

$A,B,C$
מנהלות סדרה של קרבות בזוגות. לכל אחת הסתברות קבועה לנצח ללא קשר לזהות היריבה:
\[
P(A)=0.3,\quad P(B)=1, \quad P(C)=0.5\,.
\]
מי שנפגעת לא ממשיכה להשתתף בקרבות. יורים את היריות בסדר קבוע 
$A,B,C$.
אם שתי יריבות עדיין עומדות, היורה יכולה להחליט למי היא יורה. אפשר להניח שכל אחת מקבלת החלטה מיבטיבית נגד מי לירות.

\que{1} 
מה האסטרטגיה המיטבית של
$A$?

\que{2}
נניח ש-%
$A$
יורה את היריה הראשונה באוויר. האם זו אסטרטגיה טובה יותר?

\textbf{רמז:}
חשב את ההסתברויות המותנות של 
$A$
לנצח בתלות בהחלטתה אל מי לירות קודם
$B$
או
$C$.
\end{prob}

\solution{}

סימון:
$X\stackrel{H}{\longrightarrow}Y$
מסמן ש-%
$X$
יורה לעבר
$Y$
ופוגע.
$X\stackrel{M}{\longrightarrow}Y$
מסמן ש-%
$X$
יורה לעבר
$Y$
ומחטיא.

\ans{1}
חשב את ההסתברויות המותנות ש-%
$A$
תנצח.

\textbf{מקרה 1:} 
$A$
בוחרת לירות את הירייה הראשונה לעבר 
$C$.

אם 
$A\stackrel{M}{\longrightarrow}C$
(הסתברות
$0.7$)
אזי
$B\stackrel{H}{\longrightarrow}C$
כי
$C$
היא מסוכנת יותר מ-%
$A$.
כעת
$A$
יורה שוב לעבר 
$B$
עם הסתברות
$0.3$
לפגוע, אבל אם
$A$
מחטיאה,
$B\stackrel{H}{\longrightarrow}A$
בהסתברות
$1$
ו-%
$A$
מפסידה.

אם
$A\stackrel{H}{\longrightarrow}C$
(הסתברות
$0.3$)
אזי
$B\stackrel{H}{\longrightarrow}A$
$1$
ו-%
$A$
מפסידה.

חשב את התוחלת עם הערכים
$1$
כאשר 
$A$
מנצחת ו-%
$0$
כאשר
$A$ 
מפסידה:
\vspace*{-3ex}
\[
\renewcommand*{\arraystretch}{2.5}
\begin{array}{l}
E(\textrm{מנצחת}\;A|C\;\textrm{ב- קודם לירות בוחרת}\;A) =\\
%\qquad E(A \;\textrm{wins}\;|\;A\;\textrm{misses}\;C) + E(A \;\textrm{wins}\;|\;A\;\textrm{hits}\;C)=\\
\qquad \overbrace{1\cdot (0.7\cdot 0.3)}^{A\stackrel{M}{\longrightarrow}C, A\stackrel{H}{\longrightarrow}B}+ \overbrace{0\cdot (0.7\cdot 0.7\cdot 1)}^{A\stackrel{M}{\longrightarrow}C, A\stackrel{M}{\longrightarrow}B, B\stackrel{H}{\longrightarrow}A}+ \overbrace{0\cdot (0.3\cdot 1)}^{A\stackrel{M}{\longrightarrow}C, B\stackrel{H}{\longrightarrow}A}=0.2100\,.
\end{array}
\]
\textbf{מקרה 2:} $A$
בוחרת לירות את הירייה הראשונה לעבר 
$B$.

אם
$A\stackrel{M}{\longrightarrow}B$
(הסתברות
$0.7$),
אזי כמו במקרה הקודם
$B\stackrel{H}{\longrightarrow}C$
ול-%
$A$
הזדמנות אחת נוספת לפגוע ב-%
$B$
(הסתברות
$0.3$),
אחרת
$B\stackrel{H}{\longrightarrow}A$
בהסתברות
$1$
ו-%
$A$
מפסידה.

אם
$A\stackrel{H}{\longrightarrow}B$
(הסתברות
$0.3$)
אזי
$A,C$
יורות לסירוגין אחת לעבר השנייה עד שאחת נפגעת. התסריטים האפשריים הם:
\[
\begin{array}{ll}
(1)&C\stackrel{H}{\longrightarrow}A\\
(2)&C\stackrel{M}{\longrightarrow}A \stackrel{H}{\longrightarrow}C\\
(3)&C\stackrel{M}{\longrightarrow}A \stackrel{M}{\longrightarrow}C\stackrel{H}{\longrightarrow}A\\
(4)&C\stackrel{M}{\longrightarrow}A \stackrel{M}{\longrightarrow}C\stackrel{M}{\longrightarrow}A\stackrel{H}{\longrightarrow}C\\
(5)&C\stackrel{M}{\longrightarrow}A \stackrel{M}{\longrightarrow}C\stackrel{M}{\longrightarrow}A\stackrel{M}{\longrightarrow}C\stackrel{H}{\longrightarrow}A\\
(6)&C\stackrel{M}{\longrightarrow}A \stackrel{M}{\longrightarrow}C\stackrel{M}{\longrightarrow}A\stackrel{M}{\longrightarrow}C\stackrel{M}{\longrightarrow}A\stackrel{H}{\longrightarrow}C\\
&\cdots
\end{array}
\]
ההסתברות ש-%
$A$
מנצחת כי היא פוגעת ב-%
$C$
בסופו של דבר היא סכום ההסתברויות של הסתריטים הזוגיים ברשימה:
\begin{eqnarray*}
P(\textrm{מנצחת} \;A\;| B\;\textrm{פוגעת ב-}\;A )&=&(0.5 \cdot 0.3) + \\
&&(0.5 \cdot 0.7) (0.5 \cdot 0.3) + \\
&&(0.5 \cdot 0.7) (0.5 \cdot 0.7) (0.5 \cdot 0.3)+ \cdots\\
&=&0.15 \sum_{i=0}^{\infty} 0.35^i= \frac{0.15}{1-0.35}=\frac{3}{13}\approx 0.2308\,.
\end{eqnarray*}
באופן דומה, ההסתברות ש-%
$C$
מנצחת היא
$\disfrac{0.5}{1-0.35}=\disfrac{1}{13}\approx 0.0760$.

התוחלת היא:
\vspace*{-4ex}
\[
\renewcommand*{\arraystretch}{2.5}
\begin{array}{l}
E(\textrm{מנצחת}\;A) =E(\textrm{מנצחת}\;A\;|\;B\;\textrm{ את מחטיאה }\;A) + E(\textrm{מנצחת}\;A\;|\;B\;\textrm{ ב- פוגעת}\;A)=\\
\qquad
\overbrace{1\cdot (0.7\cdot 1\cdot 0.3)}%
^{A\stackrel{M}{\longrightarrow}B, B\stackrel{H}{\longrightarrow}C, A\stackrel{H}{\longrightarrow}B}+

\overbrace{0\cdot (0.7\cdot 1\cdot 0.7\cdot 1)}%
^{A\stackrel{M}{\longrightarrow}B, B\stackrel{H}{\longrightarrow}C,A\stackrel{M}{\longrightarrow}B,B\stackrel{H}{\longrightarrow}A} +

\overbrace{1\cdot (0.2308)}%
^{A\stackrel{H}{\longrightarrow}B, C\stackrel{H}{\longleftrightarrow*}A,C\stackrel{H}{\longrightarrow}A} +
\overbrace{0\cdot (0.3\cdot (0.0769))}%
^{A\stackrel{H}{\longrightarrow}B, C\stackrel{H}{\longleftrightarrow}A,C\stackrel{H}{\longrightarrow}A}
\approx\\
\qquad 0.2792\,,
\end{array}
\]
שהיא גבוהה יותר מהתוחלת לנצח על ידי ירי תחילה לעבר 
$C$.

\ans{2}
אם
$A$
יורה לאוויר ולא פוגעת באף יריבה אזי
$B\stackrel{H}{\longrightarrow}C$
בהסתברות 
$1$
ו-%
$A$
יכולה לנסות לפגוע ב-%
$B$
בהסתברות
$0.3$.
התוחלת היא:
\[
E(\textrm{מנצחת}\;A\;|\;\textrm{באוויר יורה}\;A) = 1\cdot(0.3) + 0\cdot(0.7)=0.3\,,
\]
שהיא גבוהה יותר מהתוחלת של שתי האסטרטגיות האחרות!

\textbf{סימולציה}
\selectlanguage{english}
\begin{verbatim}
For A fires first at C:
Expectation of wins = 0.2100
Average wins        = 0.2138
For A fires first at B:
Expectation of wins = 0.2792
Average wins        = 0.2754
For A fires in the air:
Expectation of wins = 0.3000
Average wins        = 0.3000
\end{verbatim}
\selectlanguage{hebrew}

%%%%%%%%%%%%%%%%%%%%%%%%%%%%%%%%%%%%%%%%%%%%%%%%%%%%%%%%%%%%%
