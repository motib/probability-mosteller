% !TeX root = mos-he.tex

%%%%%%%%%%%%%%%%%%%%%%%%%%%%%%%%%%%%%%%%%%%%%%%%%%%%%%%%%%%%%%%%

\selectlanguage{hebrew}

\thispagestyle{empty}

\begin{center}
\textbf{\LARGE הבעיות המאתגרות בהסתברות של Mosteller}

\bigskip
\bigskip
\bigskip

\textbf{\Large מוטי בן-ארי}

\bigskip

\L{\url{http://www.weizmann.ac.il/sci-tea/benari/}}

\bigskip
\bigskip
\bigskip

 גרסה
$1.1$

\bigskip

\today

\end{center}

\vfill

\selectlanguage{english}
\begin{center}
\copyright{} Moti Ben-Ari $2022$
 \end{center}
 
\begin{small}
This work is licensed under Attribution-ShareAlike 4.0 International. To view a copy of this license, visit \url{http://creativecommons.org/licenses/by-sa/4.0/}.
\end{small}
\newpage

\selectlanguage{hebrew}

\tableofcontents

\newpage

\begin{center}
\textbf{\LARGE מבוא}
\end{center}

\addcontentsline{toc}{section}{\large מבוא}

\bigskip

\textbf{Mosteller Frederick}

\L{Frederick Mosteller}
($1916$-$2006$)
ייסד את המחלקה לסטטיסטיקה באוניברסיטת 
\L{Harvard}
והיה ראש המחלקה מ-%
$1957$
ועד
$1971$.
ל-%
\L{Mosteller}
התעניין בחינוך בסטטיסקיה וחיבר ספרי לימוד חלוציים כולל 
\L{\cite{pwsa}}
שהדגיש את הגישה ההסתברותי לסטטיסטיקה, ו-%
\L{\cite{bsda}}
שהיה אחת מספרי הלימוד הראשונים בניתוח מידע. בראיון תיאר 
\L{Mosteller}
את ההתפתחות של גישתו להוראת הסטטיסטיקה
\L{\cite{gse}}.

\medskip

\textbf{מסמך זה}

מסמך זה הוא "עיבוד" לספרו של 
\L{Mosteller}: 
\textbf{חמישים בעיות מאתגרות בהסתברות ופתרונותהן}
\L{\cite{fifty}}.
הבעיות הפתרונות מוצגות ככל האפשר בצורה נגישה לקוראים עם ידע בסיסי בהסתברות, ובעיות רבות נגישות לתלמידי תיכון ולמורים. שכתבתי אתה הבעיות והפתרונות עם חישובים מפורטים, הסברים נוספים ואיורים. לעתים כללתי פתרונות נוספים.

הבעיות שונו כדי שיהיו נגישות: פישטתי את הבעיות, חילקתי אותן לתת-בעיות והוספתי רמזים. כהעדפה אישית ניסחתי אותן מחדש בצורה מופשטת יותר מ-%
\L{Mosteller}
ולא השתמשתי ביחידות כגון אינצ'ים ומטבעות כגון דולרים. המספור והכותרות נשארו כדי להקל על השוואה עם ספרו של
\L{Mosteller}.

מחשבונים מודרניים, כולל אפליקציות לסמארטפון, מסוגלים לבצע את כל החישובים ללא קושי, ובכל זאת התשמשתי בקירוב של
\L{Stirling}.

הבעיות המסומנות ב-%
$D$
קשות יותר. אולם גם בעיות שאינן מסומנות ב-%
$D$
יכולות להיותקשה ופתרון, ולכן אל נא להתייאש אם לא תוכלו לפתור אותן. בכל זאת שווה לנסות לפתור את כולן כי כל התקדמות לקראת פתרון מעודדת.

הוספתי שני סעיפים שלא נמצאים בספר. הסעיף ראשון כולל חזרה על מושגים בסיסיים בהסתברות לפי
\L{\cite{ross}}.
בגלל שתלמידים עלולים לא להכיר מושגים כגון משתנה אקראי ותוחלת, המושגים הללו הוסברו יותר לעומק. הסעיף שני דן בבעיית מונטי הול שדומה מאוד לדילמת האסיר (בעיה%
~$13$).
אני טוען ניתן להקל על הבנתן אם מפרשים אותן בפירוש הבייסיאנית של הסתברות.

\textbf{קוד מקור}

רשיון זכויות יוצרים 
\L{CC-BY-SA}
מאפשר לקוראים להפיץ את המסמך בחופשיות ולשנות אותו כפי שמתואר מרשיון. קוד מקור ב-%
\L{\LaTeX}
ו-%
\L{Python}
ניתן למצוא ב:
\vspace{-2ex}
\begin{center}
\L{\url{https://github.com/motib/probability-mosteller/}}
\end{center}

\textbf{הבעת תודה}

אני אסיר תודה ל-%
\L{Michael Woltermann}
עבור הצעותיו המקיפות.
\L{Aaron M. Montegomery}
הסביר לי בסבלנות היבטים שונים של הילוך מקרי. 
\L{David Fortus}
העיר הערות מועילות.

\textbf{סימולציות}

סימולציות
\L{\emph{Monte Carlo}}
(על שם קזינו מופרסם במונקו) נכתבו בשפת התכנות
\L{Python 3}.
תכנית מחשב מבצעת ניסוי כגון "הטלת זוג קוביות" או "הטלת מטעה" מספר רב מאוד של פעמים, מחשב ממוצעים או יחס ההצלחות להפסדים ומציג אותם. השתמשתי במחוללי מספרים אקראיים הבנויים בתוך
\L{Python},
\L{\texttt{random.random()}}
ו-%
\L{\texttt{random.randint()}},
כדי לקבל תוצאות אקראיות לכל ניסוי.

כל תכנית מריצה סימולציה המורכת מ-%
$10000$
ניסויים והתוצאות מוצגות עם ארבע ספרות לאחר הנקודה העשרונית. כמעט תמיד התוצאה לא תהיה זהה לתוצאה שמתקבלת מחישוב ההסתברות או התוחלת. 

שמות הקבצים הם 
\L{\texttt{N-name.py}}
כאשר
\L{\texttt{N}}
הוא מספר הבעיה ו-%
\L{\texttt{name}}
הוא שם הבעיה באנגלית.

שתי תוצאות מוצגות (באנגלית) עבור כל סימולציה:
\begin{itemize}
\item
התוצאה התיאורטית שהיא 
\textbf{הסתברות}
\L{(Probability)}
או
\textbf{תוחלת}
\L{(Expectation)}.
בדרך כלל במקום להעתיק את הערכים המחושבים מהמסמך, התכנית מחשבת אותם מהנוסחאות.
\item
תוצאת הסימולציה שהיא
\textbf{היחס בין מספר ההצלחות לבין מספר הניסויים}
\L{(Proportion)}

שמקביל להסתברות, או
\textbf{ממוצע ההצלחות}
\L{(Average)}
שמקביל לתוחלת.
\end{itemize}
חשוב להבין שהסתברות ותוחלת הן מושגים תיאורטיים. 
\textbf{חוק המספרים הגדולים}
מבטיח שהתוצאות של מספר רב של ניסויים תהינה קרובות לערכים התיאורטיים, אבל הם לא יהיו זהות. למשל, ההסתברות לקבל 
$6$
בהטלת קוביה הוגנת היא
$1/6\approx 0.1667$.
בהרצת סימולציה של 
$10000$
הטלות קיבלתי טווח של ערכים:
$0.1684, 0.1693, 0.1687, 0.1665, 0.1656$.
