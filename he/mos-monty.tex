% !TeX root = mos-he.tex

%%%%%%%%%%%%%%%%%%%%%%%%%%%%%%%%%%%%%%%%%%%%%%%%%%%%%%%%%%%%%

\selectlanguage{hebrew}

\begin{center}
\textbf{\LARGE בעיית מונטי הול}
\end{center}

\addcontentsline{toc}{section}{\large בעיית מונטי הול}

%%%%%%%%%%%%%%%%%%%%%%%%%%%%%%%%%%%%%%%%%%%%%%%%%%%%%%%%%%%%%

בעיית אסיר (בעיה%
~$13$)
דומה מאוד לבעיית מונטי הול
\L{(Monty Hall)}
המפורסמת
\L{\cite{carlton,jason}}.
סעיף זה יסביר את הקשר בין שתי הבעיות ויטען שקל יותר להבין את הבעיות והפתרונות בפירוש הבייסיאנית 
\L{(bayesian)}
ולא בפירוש התדירות
\L{(frequentist)}
שלפיו נכתב ספרו של
\L{Mosteller}.

\textbf{כללי המשחק:}
בשעשועון בטלוויזיה המתחרה ניצב לפני שלוש דלתות. מאחרי כל אחת מהדלתות נמצא אחד הפרסים שהם מכונית אחת ושתי עזים. מיקום כל פרס הוא אקראי בהסתברות שווה והמנחה יודע את המיקום של כל אחד מהפרסים. המתחרה בוחר דלת אחת ולפני שמגלים למתחרה איזה פרס נמצא מאחורי הדלת שבחר, המנחה פתוח את אחת הדלתות שמאחוריהן נמצאות העזים. קיימות שתי אפשרויות:
\begin{itemize}
\item
אם המתחרה בחר בדלת עם המכונית, הבחירה של המנחה בין שתי הדלתות האחרות היא אקראית בהסתברות שווה.
\item
אם המתחרה בחר בדלת עם עז, המנחה יפתח את הדלת עם העז השנייה.
\end{itemize}
לאחר פתיחת הדלת המתחרה מחליט בין להישאר עם בחירתו המקורית לבין לשנות אותה ולבחור בדלת האחרת שלא נפתחה. המנחה פותח את הדלת שנבחרה והמתחרה זוכה בפרס שנמצא שם. מה ההסתברות שהמתחרה זוכה במכונית אם הוא נשאר עם בחירתו המקורית ומה ההסתברות שהוא זוכה במכונית אם הוא משנה אותה ובוחר בדלת האחרת?

\textbf{פתרון שגוי ופתרון נכון:} 
רבים טוענים שאין הבדל בין ההסתברויות כי המכונית נמצאת מאחר אחת משתי הדלתות הסגורות. פתרון זה שגוי כי אין כאן ניסוי בלתי-תלוי, כלומר, המנחה לא מטיל מטבע כדי להחליט מחדש לאן למקם את המכונית ולאן למקם את העז שנשאר. כל אחד מהפתרונות שבעיית האסיר ייתן את הפתרון נכון לבעיית מונטי הול שהוא שבהסתברות 
$2/3$
המכונית נמצאת מאחרי הדלת שלא נבחרה ושכדאי להחליף. הנה רשימת המאורעות וניתן לצייר עץ כמו באיור%
~\ref{f.pp}.
\begin{description}
\item[$e_1$:] 
המתחרה בוחר את הדלת עם המכונית והמארח פותח את הדלת עם עז-1.
\item[$e_2$:]
המתחרה בוחר את הדלת עם המכונית והמארח פותח את הדלת עם עז-2.
\item[$e_3$:]
המתחרה בוחר את הדלת עם עז-1 והמארח פותח את הדלת עם עז-2.
\item[$e_4$:]
המתחרה בוחר את הדלת עם עז-2 והמארח פותח את הדלת עם עז-1.
\end{description}

\textbf{הפירוש הבייסיאנית:}
לפי פירוש התדירות ברור שבסדרת משחקים אם המתחרה תמיד יחליף את בחירתו הוא יזכה בשני שליש מהמשחקים (ראו תוצאות של סימולציה). אבל כדי לגבש אסטרטגיה עבור המתחרה, עדיף לדון בבעיה בפירוש הבייסיאנית בו הסתברות היא רמת 
\textbf{האמונה}.
בשעשועון מדובר במשחק אחד עם מתחרה מסויים וביום מסויים. המידע הנוסף שמתקבל על ידי פתיחת דלת עם עז יכול להעלות את אמונתו של המתחרה על מקום הימצאותה של המכונית, להורידה או לא לשנותה כלל. הניסוחים המקובלים של הבעיה לא שואלים על הסתברויות אלא מבקשים אסטרטגיה: האם כדאי למתחרה לשנות את החלטתו או לא? ברור שהכוונה היא לפירוש בייסיאנית מבוסס אמונה. 

נניח ששינו את כללי המשחק כך שהמתחרה
\textbf{לא}
רשאי לשנות את החלטתו. ההסתברויות לא משתנות: ההסתברות לזכות במכונית נשארת
$1/3$
והסתברות לזכות בעז נשארת
$2/3$.
לא משנה איזו דלת המנחה פותח כי המתחרה לא יכול לעשות כלום. הבעיית המקורית שונה. ההסתברויות לא משתנות במובן שההסתברות שהמכונית נמצאת האחורי הדלת שנבחרה בהחלטה המקורית היא
$1/3$,
והסתברות שהיא נמצאת האחורי שתי הדלתות האחרות היא
$2/3$.
אבל כעת ההסתברות
$2/3$
כאילו "מרוכזת" בדלת שהמנחה לא פתח, ולכן, לפי הפירוש הבייסיאנית, אמונתו של המתחרה שהמכונית נמצאת מאחרי הדלת השנייה מוכפלת מ-%
$1/3$
ל-%
$2/3$
וכדאי לשנות את ההחלטה.

גם את בעיית האסיר עדיף לפרש לפי הפירוש הבייסיאנית. מדובר באסירים מסויימים ביום מסויים. השאלה שואלת איך 
\textbf{אמונתו}
של 
$A$
שהוא עתיד להשתחרר משתנה לאור המידע הנוסף ש-%
$B$
ישוחרר. הפרתונות מראים שאין הצדקה לשנות את אמונתו. הפירוש הבייסיאנית סמוי בניסוח של
\L{Mosteller}:
\begin{quote}
[$A$]
חושב שאם מפקד הכלא אומר "%
$B$
ישוחרר", הסיכויים שלו ירדו ל-%
$1/2$ \ldots{}
\end{quote}
מהמלים "חושב" ו-"שלו" ברור שהכוונה היא לפירוש הבייסיאנית, כי בפירוש התדירות, ההסתברות היא אובייקטיבית ולא מה שאתה חושב, וההסתברות לא שייכית לאף אחד.

\medskip

\sml{}

\selectlanguage{english}
\begin{verbatim}
Wins when staying with original door = 0.3359
Wins when changing door              = 0.6641
\end{verbatim}
\selectlanguage{hebrew}

